%%%%%%%%%%%%%%%%%%%%%%%%%%%%%%%%%%%%%%%%%%%%%%%%%%%%%%%%%%%%%%%%%%%%%%%%%%%%%%%%
%                                                                              %
% BlackArch Linux Guide                                                        %
%                                                                              %
%%%%%%%%%%%%%%%%%%%%%%%%%%%%%%%%%%%%%%%%%%%%%%%%%%%%%%%%%%%%%%%%%%%%%%%%%%%%%%%%

\documentclass[a4paper, oneside, 11pt]{book}

%%% INCLUDES %%%
\renewcommand{\familydefault}{\sfdefault}

\usepackage{array}
\usepackage{color}
\usepackage{enumerate}
\usepackage{fancyhdr}
\usepackage{fancyvrb}
\usepackage{geometry}
\usepackage{graphicx}
\usepackage{html}
\usepackage{hyperref}
\usepackage{ifpdf}
\usepackage{listings}
\usepackage{pstricks}
\usepackage{supertabular}
\usepackage{tocloft}
\usepackage[utf8]{inputenc}

\usepackage[spanish]{babel}
\usepackage[utf8]{inputenc}
\usepackage[T1]{fontenc}

%%% LAYOUT %%%
\setlength{\parindent}{0em}
\setlength{\parskip}{1.5ex plus0.5ex minus0.5ex}
\geometry{left=2.5cm, textwidth=16cm, top=3cm, textheight=25cm, bottom=3cm}
\widowpenalty=2000
\clubpenalty=1000
\frenchspacing
\sloppy
\pagecolor[HTML]{FFFFFF}
\color[HTML]{000000}
\setcounter{tocdepth}{10}
\setcounter{secnumdepth}{10}

\hypersetup{
  pdftitle={BlackArch Linux, Gu{\'i}a de BlackArch Linux},
  pdfsubject={BlackArch Linux, Guía de BlackArch Linux},
  pdfauthor={BlackArch Linux, BlackArch Linux},
  pdfkeywords={BlackArch Linux, Penetration Testing, Security, Hacking, Linux, Pruebas de penetración, Seguridad},
  pdfcenterwindow=true,
  colorlinks=true,
  breaklinks=true,
  linkcolor=red,
  menucolor=red,
  urlcolor=red
}

% syntax highlighting
% all options should be set here document wide
\lstset{
backgroundcolor=\color[HTML]{EEEEEE},
frame=single,
basicstyle=\footnotesize\ttfamily,
float,
deletekeywords={return},
otherkeywords={mkdir, curl, sudo, sha1sum, grep, cut, sort, wget, makepkg,
pacman, blackman},
keywordstyle=\color{orange},
commentstyle=\color{blue},
stringstyle=\color{red},
language=bash,
showspaces=false,
showtabs=false,
tabsize=2,
literate={á}{{\'a}}1 {é}{{\'e}}1 {í}{{{\'\i}}}1 {ó}{{\'o}}1 {ö}{{\"o}}1 {ő}{{\H o}}1 {ú}{{\'u}}1 {Ú}{{\'U}}1 {ü}{{\"u}}1 {ű}{{\H u}}1 {Ü}{{\"U}}1
}

%%% HEADER / FOOTER %%%
\setlength{\headheight}{33pt}
\setlength{\headsep}{33pt}
\lhead{{\includegraphics[width=1cm,height=1cm]{images/logo.png}}}
\rhead{Guía de BlackArch Linux}

%%% CUSTOM MACROS %%%
% for html links
\ifpdf\else
\def\href#1#2{\htmladdnormallink{#2}{#1}}
\fi

%------------------%
%  TITLE PAGE      %
%------------------%
\begin{document}
\pagestyle{empty}
\begin{center}
\begin{figure}[htbp]
\centering
\vspace{0.5cm}
\includegraphics[width=8cm]{images/logo.png}
\label{fig:logo}
\end{figure}
\vspace{0.5cm}
\Huge{\textbf{Guía de BlackArch Linux}}\\
\vspace{1cm}
\Large{\color{red}https://www.blackarch.org/}\\
\vspace{0.5cm}
\end{center}
\newpage
\tableofcontents
\newpage
\pagestyle{fancy}

%------------------%
%  Chapter 1       %
%------------------%

\chapter{Introducción}

\section{Resumen}
La guía de BlackArch Linux está divida en varias partes:
\begin{itemize}
\item Introducción - Proporciona una visión general, introducción e información adicional útil del proyecto
\item Guía de usuario - Todo lo que un usuario típico necesita saber para usar eficazmente BlakArch
\item Guía del desarrollador - Cómo empezar a desarrollar y contribuir para BlackArch
\item Guía de herramientas - Uso de herramientas detalladas en profundidad junto con ejemplos (WIP)
\end{itemize}

\section{\mbox{?`}Qué es BlackArch Linux?}
BlackArch es una distribución Linux para pruebas de penetración e investigación en seguridad.
Está basada en \href{https://www.archlinux.org/}{ArchLinux} y los usuarios pueden instalar los componentes de BlackArch individualmente o en grupos directamente sobre ella.

El conjunto de herramientas está repartido tal como en un 
\href{https://wiki.archlinux.org/index.php/Unofficial\_User\_Repositories}
{repositorio de usuarios no oficial} de Arch Linux, así que puedes instalar BlackArch sobre una
instalación de Arch Linux existente. Los paquetes pueden ser instalados individualmente o por
categoría.

La constante expansión de repositorios incluye cerca de \href{https://www.blackarch.org/tools.html}{1500} herramientas.
Todas las herramientas son probadas a fondo antes de ser agregadas al código base para mantener la calidad del repositorio.
% should quickly describe the testing methods/code review procedures etc

\section{Historia de BlackArch Linux}
Pronto...

\section{Plataformas soportadas}
Pronto...

\section{Involúcrate}
Puedes contactarte con el equipo de BlackArch usando los siguientes métodos:

Sitio web: \url{https://www.blackarch.org/}

Mail: \href{mailto:team@blackarch.org}{team@blackarch.org}

IRC: \url{irc://irc.freenode.net/blackarch}

Twitter: \url{https://twitter.com/blackarchlinux}

Github: \url{https://github.com/Blackarch/}

%------------------%
%  Chapter 2       %
%------------------%


\chapter{Guía de usuario}

\section{Instalación}
Las siguientes secciones te mostrarán cómo configurar el repositorio de BlackArch e
instalar paquetes. BlackArch soporta tanto instalación desde repositorios usando
paquetes binarios como compilar e instalar desde fuentes.

BlackArch es compatible con instalaciones normales de Arch. Atúa como un repositorio
de usuario no oficial. Sin embargo, si quieres una ISO, mira la sección de 
\href{https://www.blackarch.org/downloads.html#iso}{Live ISO}.

\subsection{Instalación sobre ArchLinux}
Ejecuta \href{https://blackarch.org/strap.sh}{strap.sh} como administrador (root) y sigue las
instrucciones. Mira el siguiente ejemplo.

\begin{lstlisting}
   curl -O https://blackarch.org/strap.sh
   sha1sum strap.sh # debería coincidir: 86eb4efb68918dbfdd1e22862a48fda20a8145ff
   sudo ./strap.sh
\end{lstlisting}

Ahora descarga una copia reciente de la lista de paquetes maestros y sincroniza los paquetes:
\begin{lstlisting}
  sudo pacman -Syyu
\end{lstlisting}


\subsection{Instalación de paquetes}
Ahora debe ser posible instalar las herramientas desde el repositorio de BlackArch.
\begin{enumerate}
\item Para listar todas las herramientas disponibles, ejecuta:
\begin{lstlisting}
  pacman -Sgg | grep blackarch | cut -d' ' -f2 | sort -u
\end{lstlisting}

\item Para instalar todas las herramientas, ejecuta:
\begin{lstlisting}
  pacman -S blackarch
\end{lstlisting}

\item Para instalar una categoría de herramientas, ejecuta:
\begin{lstlisting}
  pacman -S blackarch-<categoría>
\end{lstlisting}

\item Para ver las categorías de BlackArch, ejecuta:
\begin{lstlisting}
  pacman -Sg | grep blackarch
\end{lstlisting}

\end{enumerate}

\subsection{Instalación de paquetes desde fuentes}
Como parte de un método alternativo de instalación, puedes construir los 
paquetes de BlackArch desde fuentes. Puedes encontrar los PKGBUILDs en
\href{https://github.com/BlackArch/blackarch/tree/master/packages}{github}. para
construir el repositorio entero, puedes usar la herramienta
\href{https://github.com/BlackArch/blackman}{Blackman}.
\begin{itemize}
\item Primero debes instalar Blackman. Si el repositorio de paquetes de BlackArch
está configurado en tu máquina, puedes instalar Blackman:
\begin{lstlisting}
  pacman -S blackman
\end{lstlisting}

\item Puedes construir e instalar Blackman desde su fuente:
\begin{lstlisting}
  mkdir blackman
  cd blackman
  wget https://raw2.github.com/BlackArch/blackarch/master/packages/blackman/PKGBUILD
  # Asegúrate de que el PKGBUILD no ha sido maliciosamente manipulado con:
  makepkg -s
\end{lstlisting}

\item También puedes instalar Blackman desde el AUR:
\begin{lstlisting}
  <cliente ayudante de AUR que usas> -S blackman
\end{lstlisting}

\end{itemize}

\subsection{Uso básico de Blackman} 
Blackman es muy fácil de usar, aunque las banderas (flags) son diferentes de lo que típicamente
esperarías de algo como pacman. Su uso básico se describe a continuación
\begin{itemize}
\item Descarga, compila e instala paquetes:
\begin{lstlisting}
  sudo blackman -i package
\end{lstlisting}

\item Descarga, compila e instala categorías completas:
\begin{lstlisting}
  sudo blackman -g group
\end{lstlisting}

\item Descarga, compila e instala todas las herramientas de BlackArch:
\begin{lstlisting}
  sudo blackman -a
\end{lstlisting}

\item Para listar las categorías de BlackArch:
\begin{lstlisting}
  blackman -l
\end{lstlisting}

\item Para listar herramientas de una categoría:
\begin{lstlisting}
  blackman -p category
\end{lstlisting}

\end{itemize}

\subsection{Instalación desde live, netinstall, ISO o sobre Arch Linux}
Puedes instalar BlackArch Linux desde una de nuestras ISOs live o netinstall.\\Consulta
\url{https://www.blackarch.org/download.html#iso}. Los siguientes pasos se
requieren después de que ha iniciado la ISO.
\begin{itemize}
\item Instalar el paquete blackarch-installer:
\begin{lstlisting}
  sudo pacman -S blackarch-installer
\end{lstlisting}

\item Ejecuta:
\begin{lstlisting}
  sudo blackarch-install
\end{lstlisting}

\end{itemize}

%------------------%
%  Chapter 3       %
%------------------%

\chapter{Guía del desarrollador}

\section{Construcción del sistema y Repositorios de Arch}

Los archivos PKGBUILD son scripts de construcción (build). Cada uno le dice a makepkg cómo crear un
paquete. Los archivos PKGBUILD son escritos en Bash.

Para más información, lee (u ojea) lo siguiente:
\begin{itemize}
\item \href{https://wiki.archlinux.org/index.php/Creating_Packages_(Espa%C3%B1ol)}{Arch Wiki: Creación de paquetes}
\item \href{https://wiki.archlinux.org/index.php/Makepkg_(Espa%C3%B1ol)}{Arch Wiki: Makepkg}
\item \href{https://wiki.archlinux.org/index.php/PKGBUILD_(Espa%C3%B1ol)}{Arch Wiki: PKGBUILD}
\item \href{https://wiki.archlinux.org/index.php/Arch_packaging_standards_(Espa%C3%B1ol)}{Arch Wiki: Estándares de paquetación Arch}
\end{itemize}

\section{Estándares de Blackarch PKGBUILD}
Por el bien de la simplicidad, nuestros PKGBUILDs son similares a los del AUR,
con algunas pequeñas diferencias descritas a continuación. Cada paquete debe
pertenecer mínimo a BlackArch, habrá también muchas coincidencias con
múltiples paquetes pertenecientes a múltiples grupos.

\subsection{Grupos}
Para permitir a los usuarios instalar rápida y fácilmente un rango específico de paquetes,
los paquetes deben estar separados en grupos. Los grupos permiten a los usuarios simplemente
ejecutar "pacman -S <nombre del grupo>" para arrastrar muchos paquetes.

\subsubsection{blackarch}
El grupo blackarch es el grupo base al que todos los paquetes deben pertenecer. Esto permite
a los usuarios instalar cada paquete fácilmente.

Lo que debería estar aquí: Todo.

\subsubsection{blackarch-anti-forensic}
Los paquetes que son usados para contrarrestar las actividades forenses,
incluyendo el cifrado, la esteganografía y todo lo que modifica los atributos de los archivos.
Todo esto incluye herramientas para trabajar con cualquier cosa, en general, que hace cambios a un sistema
con el propósito de ocultar información.

Ejemplos: luks, TrueCrypt, Timestomp, dd, ropeadope, secure-delete

\subsubsection{blackarch-automation}
Paquetes que son usados como herramienta o automatización de flujo de trabajo.

Ejemplos: blueranger, tiger, wiffy

\subsubsection{blackarch-backdoor}
Paquetes que explotan o abren puertas traseras en sistemas que ya son vulnerables.

Ejemplos: backdoor-factory, rrs, weevely

\subsubsection{blackarch-binary}
Paquetes que operan sobre archivos binarios de alguna forma.

Ejemplos: binwally, packerid

\subsubsection{blackarch-bluetooth}
Paquetes que explotan cualquier cosa referente al estándar Bluetooth (802.15.1).

Ejemplos: ubertooth, tbear, redfang

\subsubsection{blackarch-code-audit}
Paquetes que revisan código fuente existente para análisis de vulnerabilidades.

Ejemplos: flawfinder, pscan

\subsubsection{blackarch-cracker}
Paquetes usados para crackear funciones criptográficas, por ejemplo: hashes.

Ejemplos: hashcat, john, crunch

\subsubsection{blackarch-crypto}
Paquetes que trabajan con criptografía, exceptuando el cracking.

Ejemplos: ciphertest, xortool, sbd

\subsubsection{blackarch-database}
Paquetes que involucran explotación de bases de datos en cualquier nivel.

Ejemplos: metacoretex, blindsql

\subsubsection{blackarch-debugger}
Paquetes que permiten al usuario ver que está "haciendo" un programa en particular en tiempo real.

Ejemplos: radare2, shellnoob

\subsubsection{blackarch-decompiler}
Paquetes que intentan revertir un programa compilado hacia su código fuente.

Ejemplos: flasm, jd-gui

\subsubsection{blackarch-defensive}
Paquetes que son usados para protejer un usuario de malware y ataques de otros usuarios.

Ejemplos: arpon, chkrootkit, sniffjoke

\subsubsection{blackarch-disassembler}
Este es similar a blackarch-decompiler y habrá, probablemente, muchos
programas que asisten ambos grupos, sin embargo estos paquetes producen una salida
de ensanblador antes que el código fuente neto.

Ejemplos: inguma, radare2

\subsubsection{blackarch-dos}
Paquetes que hacen uso de ataques DoS (Denegación de servicio).

Ejemplos: 42zip, nkiller2

\subsubsection{blackarch-drone}
Paquetes que son usados para gestionar la ingeniería física de drones.

Ejemplos: meshdeck, skyjack

\subsubsection{blackarch-exploitation}
Paquetes que toman ventaja de explosiones en otros programas o servicios.

Ejemplos: armitage, metasploit, zarp

\subsubsection{blackarch-fingerprint}
Paquetes que explotan equipo biométrico de huellas digitales.

Ejemplos: dns-map, p0f, httprint

\subsubsection{blackarch-firmware}
Paquetes que explotan vulnerabilidades en firmware.

Ejemplos: firmwalker, uefi-firmware-parser.

\subsubsection{blackarch-forensic}
Paquetes que son usados para encontrar datos en discos físicos o memorias integradas.

Ejemplos: aesfix, nfex, wyd

\subsubsection{blackarch-fuzzer}
Paquetes que usan el principio de la prueba de palusa, por ejemplo:
"soltar" entradas aleatorias en el sujeto para ver que pasa.

Ejemplos: msf, mdk3, wfuzz

\subsubsection{blackarch-hardware}
Paquetes que explotan o gestionan cualquier cosa que tenga que ver
con Hardware físico.

Ejemplos: arduino, smali

\subsubsection{blackarch-honeypot}
Paquetes que actúan como "señuelo", por ejemplo, programas que parecen ser
servicios vulnerables usados para atraer hackers hacia una trampa.

Ejemplos: artillery, bluepot, wifi-honey

\subsubsection{blackarch-keylogger}
Paquetes que graban y guardan las pulsaciones del teclado en otro sistema.

Ejemplos: logkeys, xspy

\subsubsection{blackarch-malware}
Paquetes que calculan la cantidad de cualquier tipo de Software malicioso o 
detector de malware.

Ejemplos: malwaredetect, peepdf, yara

\subsubsection{blackarch-misc}
Paquetes que no encajan en ninguna categoría en particular.

Ejemplos: oh-my-zsh-git, winexe, stompy

\subsubsection{blackarch-mobile}
Paquetes que manipulan plataformas móviles.

Ejemplos: android-sdk-platform-tools, android-udev-rules

\subsubsection{blackarch-networking}
Paquetes que involucran redes IP.

Ejemplos: Casi cualquier cosa.

\subsubsection{blackarch-nfc}
Paquetes que usan el sistema NFC (Near-Field Communications).

Ejemplos: nfcutils

\subsubsection{blackarch-packer}
Paquetes que operan sobre o involucran empaquetadores.

\textit{Empaquetadores son programas que incrustan malware entre otros ejecutables.}

Ejemplos: packerid

\subsubsection{blackarch-proxy}
Paquetes que actúan como un proxy, por ejemplo, redireccionar tráfico
a través de otro nodo en Internet.

Ejemplos: burpsuite, ratproxy, sslnuke

\subsubsection{blackarch-recon}
Paquetes que buscan activamente explosiones vulnerables en lo 
salvaje. Más como un grupo sombrilla para paquetes similares.

Ejemplos: canri, dnsrecon, netmask

\subsubsection{blackarch-reversing}
Este es un grupo sombrilla para cualquier decompilador,
desensamblador o cualquier programa similar.

Ejemplos: capstone, radare2, zerowine

\subsubsection{blackarch-scanner}
Paquetes que escanean los sistemas seleccionados en busca de vulnerabilidades.

Ejemplos: scanssh, tiger, zmap

\subsubsection{blackarch-sniffer}
Paquetes que involucran análisis del tráfico de la red.

Ejemplos: hexinject, pytactle, xspy

\subsubsection{blackarch-social}
Paquetes que, ante todo, atacan sitios de redes sociales.

Ejemplos: jigsaw, websploit

\subsubsection{blackarch-spoof}
Paquetes que intentan suplatar el atacante, tal que el atacante
no se muestra como un atacante hacia la victima.

Ejemplos: arpoison, lans, netcommander

\subsubsection{blackarch-tunnel}
Paquetes que son usados para desviar el tráfico de red
en una red dada.

Ejemplos: ctunnel, iodine, ptunnel

\subsubsection{blackarch-unpacker}
Paquetes que son usados para extraer malware pre-empaquetado desde un
ejecutable.

Ejemplos: js-beautify

\subsubsection{blackarch-voip}
Paquetes que operan en programas y protocolos VOIP.

Ejemplos: iaxflood, rtp-flood, teardown

\subsubsection{blackarch-webapp}
Paquetes que operan en aplicaciones de cara a Internet.

Ejemplos: metoscan, whatweb, zaproxy

\subsubsection{blackarch-windows}
Este grupo es para cualquier paquete nativo de Windows que corre mediante Wine.

Ejemplos: 3proxy-win32, pwdump, winexe

\subsubsection{blackarch-wireless}
Paquetes que operan en redes inalambricas en cualquier nivel.

Ejemplos: airpwn, mdk3, wiffy

\section{Estructura del repositorio}
Puedes encontrar el repositorio git principal de BlackArch aquí:
\href{https://github.com/BlackArch/blackarch}{https://github.com/BlackArch/blackarch}.
También hay varios repositorios secundarios aquí:
\href{https://github.com/BlackArch}{https://github.com/BlackArch}.

Dentro del repositorio git principal, hay tres directorios importantes:

\begin{itemize}
\item docs - Documentación.
\item packages - Archivos PKGBUILD.
\item scripts - Pequeños scripts útiles.
\end{itemize}

\subsection{Scripts}
Aquí hay una referencia para el directorio de \verb|scripts/|:

\begin{itemize}
\item baaur - Este carga paquetes al AUR.
\item babuild - Construye (build) un paquete.
\item bachroot - Administra un chroot para pruebas.
\item baclean - Limpia viejos archivos .pkg.tar.xz del repositorio de paquetes.
\item baconflict - Pronto reemplazará \verb|scripts/conflicts|.
\item bad-files - Busca archivos dañados en los paquetes contruidos.
\item balock - Obtiene o libera el candado del repositorio del paquete.
\item banotify - Notifica IRC sobre pushes de paquetes.
\item barelease - Libera paquetes hacia el repositorio de paquetes.
\item baright - Imprime la información de Copyright de BlackArch.
\item basign - Firma de paquetes.
\item basign-key - Firma de una llave.
\item blackman - Este se comporta estilo pacman, pero construye desde git (no se debe confundir con 
    NRZ de Blackman).
\item check-groups - Revisa grupos.
\item checkpkgs - Revisa paquetes por errores.
\item conflicts - Revisa conflictos de archivo.
\item dbmod - Modifica una base de datos de un paquete.
\item depth-list - Crea una lista ordenada por profundidad de dependencia.
\item deptree - Crea un árbol de dependencia, listando sólo paquetes proporcionados por blackarch.
\item get-blackarch-deps - Obtiene una lista de dependencias de blackarch para un paquete.
\item get-official - Obtiene los paquetes oficiales para el lanzamiento.
\item list-loose-packages - Lista los paquetes que no están en grupos y no son
    dependencias de otros paquetes.
\item list-needed - Lista dependencias faltantes.
\item list-removed - Lista los paquetes que están en el repositorio del paquete pero no en el git.
\item list-tools - Lista las herramientas.
\item outdated - Busca paquetes que están desactualizados en el repositorio del paquete
    relativo al repositorio git.
\item pkgmod - Modifica un paquete construido.
\item pkgrel - Incrementa pkgrel en un paquete.
\item prep - Limpia el estilo del archivo PKGBUILD y encuentra errores.
\item sitesync - Sincroniza una copia local del repositorio del paquete y una copia remota.
\item size-hunt - Caza grandes paquetes.
\item source-backup - Realiza una copia de seguridad de los archivos fuente del paquete.
\end{itemize}

\section{Contribuir al repositorio}
Ésta sección te muestra como contribuir al proyecto de BlackArch Linux. 
Aceptamos peticiones pull de todos los tamaños, desde pequeños errores de tipografía
hasta nuevos paquetes. \\Para ayuda, sugerencias o preguntas, no dudes en ponerte en contacto con nosotros.
\\\\
Todo el mundo es bienvenido a contribuir. Todas las contribuciones son apreciadas.

\subsection{Tutoriales necesarios}
Por favor lee los siguientes tutoriales antes de contribuir:
\begin{itemize}
\item \href{https://wiki.archlinux.org/index.php/Arch_packaging_standards_(Espa%C3%B1ol)}{Arch Wiki: Estándares de paquetación Arch}
\item \href{https://wiki.archlinux.org/index.php/Creating_Packages_(Espa%C3%B1ol)}{Arch Wiki: Creación de paquetes}
\item \href{https://wiki.archlinux.org/index.php/PKGBUILD_(Espa%C3%B1ol)}{Arch Wiki: PKGBUILD}
\item \href{https://wiki.archlinux.org/index.php/Makepkg_(Espa%C3%B1ol)}{Arch Wiki: Makepkg}
\end{itemize}

\subsection{Pasos para contribuir}
Con el fin de presentar tus cambios al proyecto de BlackArch Linux, sigue estos 
pasos:
\begin{enumerate}
\item Haz Fork del repositorio en
\url{https://github.com/BlackArch/blackarch}
\item Realiza los cambios necesarios, (por ejemplo: PKGBUILD, archivos .patch, etc).
\item Haz commit a tus cambios.
\item Haz push de tus cambios.
\item Pídenos que unamos tus cambios, preferiblemente mediante una petición pull.
\end{enumerate}

\subsection{Ejemplo}
El siguiente ejemplo muestra una entrega de un nuevo paquete al proyecto de
BlackArch. Usamos \href{https://wiki.archlinux.org/index.php/yaourt}{yaourt}
(también puedes usar pacaur) para obtener un archivo PKGBUILD existente para
\textbf{nfsshell} desde el \href{https://aur.archlinux.org/}{AUR} y lo ajustamos
de acuerdo a nuestras necesidades.

\subsubsection{Obtener PKGBUILD}
Obtener el archivo \textit{PKGBUILD} usando yaourt o pacaur:
\begin{lstlisting}
  user@blackarchlinux $ yaourt -G nfsshell
  ==> Download nfsshell sources
  x LICENSE
  x PKGBUILD
  x gcc.patch
  user@blackarchlinux $ cd nfsshell/
\end{lstlisting}

\subsubsection{Limpiar el PKGBUILD}
Limpiar el archivo \textit{PKGBUILD} y ahorrar algo de tiempo:
\begin{lstlisting}
  user@blackarchlinux nfsshell $ ./blarckarch/scripts/prep PKGBUILD
  cleaning 'PKGBUILD'...
  expanding tabs...
  removing vim modeline...
  removing id comment...
  removing contributor and maintainer comments...
  squeezing extra blank lines...
  removing '|| return'...
  removing leading blank line...
  removing $pkgname...
  removing trailing whitespace...
\end{lstlisting}

\subsubsection{Ajustar el PKGBUILD}
Ajustar el archivo \textit{PKGBUILD}
\begin{lstlisting}
  user@blackarchlinux nfsshell $ vi PKGBUILD
\end{lstlisting}

\subsubsection{Construir el paquete}
Construir el paquete:
\begin{lstlisting}
user@blackarchlinux nfsshell $ makepkg -sf
==> Making package: nfsshell 19980519-1 (Mon Dec  2 17:23:51 CET 2013)
==> Checking runtime dependencies...
==> Checking buildtime dependencies...
==> Retrieving sources...
-> Downloading nfsshell.tar.gz...
% Total    % Received % Xferd  Average Speed   Time    Time     Time
CurrentDload  Upload   Total   Spent    Left  Speed100 29213  100 29213    0
0  48150      0 --:--:-- --:--:-- --:--:-- 48206
-> Found gcc.patch
-> Found LICENSE
...
<muchos procesos de salida de construcción y compilado aqui>
...
==> Leaving fakeroot environment.
==> Finished making: nfsshell 19980519-1 (Mon Dec  2 17:23:53 CET 2013)
\end{lstlisting}

\subsubsection{Instalar y probar el paquete}
Instalar y probar el paquete:
\begin{lstlisting}
  user@blackarchlinux nfsshell $ pacman -U nfsshell-19980519-1-x86_64.pkg.tar.xz
  user@blackarchlinux nfsshell $ nfsshell # probarlo
\end{lstlisting}

\subsubsection{Agregar, hacer commit y push al paquete}
Agregar, hacer commit y push al paquete
\begin{lstlisting}
user@blackarchlinux nfsshell $ cd /blackarchlinux/packages
user@blackarchlinux ~/blackarchlinux/packages $ mv ~/nfsshell .
user@blackarchlinux ~/blackarchlinux/packages $ git commit -am nfsshell && git push
\end{lstlisting}

\subsubsection{Crear una petición pull}
Crear una petición pull en \href{https://github.com/}{github.com}
\begin{lstlisting}
  firefox https://github.com/<contribuidor>/blackarchlinux
\end{lstlisting}

\subsubsection{Agregar remotamente}
Algo inteligente de hacer si estás trabajando remotamente y en un fork, es hacer pull desde tu propio fork y agregar el repositorio principal como uno remoto:
\begin{lstlisting}
  user@blackarchlinux ~/blackarchlinux $ git remote -v
  origin <el enlace de tu fork> (fetch)
  origin <el enlace de tu fork> (push)
  user@blackarchlinux ~/blackarchlinux $ git remote add upstream https://github.com/blackarch/blackarch
  user@blackarchlinux ~/blackarchlinux $ git remote -v
  origin <el enlace de tu fork> (fetch)
  origin <el enlace de tu fork> (push)
  upstream https://github.com/blackarch/blackarch (fetch)
  upstream https://github.com/blackarch/blackarch (push)
\end{lstlisting}

Por defecto, git debería hacer push directo al origen, pero asegúrate de que la
configuración de tu git está configurada correctamente. 
Esto no será un problema si tienges permisos para realizar el commit, 
ya que no podrás hacer push remotamente sin los permisos.

Si tienes la habilidad de hacer commit, podrías tener mas éxito usando
\textit{git@github.com:blackarch/blackarch.git} pero depende de ti.

\subsection{Peticiones}
\begin{enumerate}
\item No agregues comentarios de \textbf{Contribuidor} a los archivos de
\textit{PKGBUILD}. Agrega los nombres de los contribuidores a la sección
AUTHORS de la guía de BlackArch.
\item Por el bien de la consistencia, por favor sigue el estilo general de los otros
archivos \textit{PKGBUILD} en el repositorio y usa sangría de dos espacios.
\end{enumerate}

\subsection{Consejos generales}
\href{http://wiki.archlinux.org/index.php/Namcap}{Namcap} puede verificar los paquetes con errores.

%------------------%
%  Chapter 4       %
%------------------%

\chapter{Guía de herramientas}
Pronto...

\section{Pronto}
Pronto...

%%% APPENDIX %%%
\appendix
\include{appendix}

\end{document}

%%% EOF %%%
